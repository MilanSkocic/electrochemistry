\section{Introduction}

In the course of the last 30 years, photoelectrochemical techniques have been 
shown to be useful tools for characterizing oxidation layers. 
Interdisciplinary theoretical underpinnings were built 
\citep{morrison1980, vijh1969, stimming1986, diquarto1997, wouters2007} 
such as the Gärtne-Butler model \citep{gartner1959,butler1977} which has been 
proven to be a simple and robust model for the photocurrent generation. 
Technical progresses were achieved, allowing to study oxide layers at 
macroscopic, mesoscopic, and microscopic scales 
\citep{benaboud2007, srisrual2011}, or in-situ in high temperature corrosion 
conditions \citep{bojinov2002,skocic2016}.

Up to now, for complex oxide scales formed of several p-type and n-type phases, 
the complete description of the photocurrent energy spectra could not be achieved, 
and only semi-quantitative and/or partial information could be obtained on the 
nature of the phases present in the oxide layers. Recently, a new approach was 
proposed by \citet{petit2013} to analyze the photocurrent spectra which was 
applied to oxidized duplex stainless steels [9] as well as Ni-based and 
Zr-based alloys oxidized in LWR conditions \citep{skocic2016}. 
The numerical fitting procedure allowed to obtain high quality fits of the 
experimental data. Nonetheless, the estimation of the confidence intervals 
was not implemented.

This paper presents the additional work carried out in order to implement the 
estimation of the confidence intervals based on the fitting procedure 
developed by \citet{petit2013}. The latter was rewritten in Python which is 
an open source interpreted programming language. It is largely used in the 
scientific community \citep{langtangen2012, millman2011, kiusalaas2010, oliphant2007} 
and it comes with optimized libraries for numerical computations 
\citep{vanderwalt2011,jones2020} and a high quality 2D visualization library \citep{hunter2007}.
